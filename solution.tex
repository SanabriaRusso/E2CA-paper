In an overcrowded CSMA/ECA ($N>N_{max}$), nodes will double $CW(k)$ after collisions and reset it ($CW(k)=CW_{min}$) upon each transmission success, augmenting the collision probability. This accounts for the throughput reduction in Figure~\ref{fig:throughput}. To make it possible to achieve the collision-free state when $N>N_{max}$, we propose that nodes in CSMA/ECA do not reset $CW(k)$ after successful transmissions. This is called \emph{resiliency} from here forward.

Resiliency forces nodes to \emph{stick} to the value of the current backoff stage, $k$; resulting in a larger $CW(k)$. This measure leads to a collision-free state while $N\leq N_{max}$.

Having a greater collision-free constraint ($N_{max}$) means that more nodes are able to achieve a collision-free state. Nevertheless, in a $N\leq N_{max}$ scenario, contenders may have different deterministic backoff counters which provoke some nodes to access the channel more often than others. This fairness issue is averted with \emph{fair-share}.

Fair-share consist on allowing each contender to send $2^{k}$ packets every time its backoff expires ($B=0$), making sure that contenders with longer backoff are compensated proportionally.

Figure~\ref{fig:fairShare}, depicts how CSMA/ECA with resiliency and fair-share achieves greater throughput than CSMA/CA, maintaining a collision-free state and being fair (Jain's Fairness Index~\cite{JFI}~(JFI) equals 1) for any number of contenders.

\begin{figure}[htbp]
  \centering
  \includegraphics[width=0.7\linewidth, angle = -90]{figures/throughput/CSMA-E2CA_w_fairShare.eps}
  \caption{Throughput and Jain's Fairness Index when implementing fair-share in CSMA/ECA
  \label{fig:fairShare}}
\end{figure}

The concept fair-share, was first introduced by Fang et al. Available at SpringerLink~\cite{L_MAC2}. This work evaluates the performance of CSMA/ECA when implementing the concept in a customized C++ simulator.

%In Figure~\ref{fig:fairShare} it is appreciated through the estimation of the Jain's Fairness Index~\cite{JFI} that by implementing Fair Share every contender receives almost the same service time, therefore the system is considered fair.

\subsection*{Evaluation}
CSMA/ECA preserves backward compatibility with CSMA/CA (details in~\cite{CSMA_ECA, HE}), which is paramount for the coexistence and progressive adoption of the protocol. Many other performance evaluations, like a semi-analytical framework modelling the enhanced collision avoidance mechanism and comparing it with other access schemes (like Basic Access and RTS/CTS), are provided in~\cite{E2CA_performance}. Nevertheless, to the best of our knowledge this is the first evaluation of resilience and fair-share in CSMA/ECA.

Implementation is performed on a customized version of the COST~\cite{COST}~simulator. The system was set to be under saturation (nodes always have packets to transmit) during a period of a hundred thousand seconds at a maximum throughput of $11$Mbps. The number of contenders ranges from $2$ to $50$. Further MAC-related parameters can be found under~\emph{stats/stats.h}~in~\cite{sim:parameters}; as well as the code for the whole CSMA/ECA implementation.

Figure~\ref{fig:throughput} and Figure~\ref{fig:fairShare} are results derived from the evaluation platform.