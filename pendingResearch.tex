%CSMA/ECA is the basic idea, but there are many issues to investigate in the future. One of them, is to make it adaptive to a variable number of nodes without fairness between nodes. Open topics include how to provide traffic differentiation (i.e. Quality of Service) in top of CSMA/ECA, or how to adapt novel features such as Multi-packet Transmission / Reception and channel bonding on top of it. 

%As it is shown in~\cite{E2CA_performance}, the combination of CSMA/ECA with the Auto Rate Fallback mechanism to select the transmission rates provides a huge gain in the network performance as, once collisions are removed, they do not interfere with the ARF operation. If similar gains can be obtained when combining it with the previously mentioned mechanisms are still open challenges.

To produce a throughout analysis of CSMA/ECA, more evaluations need to be carried out under non-saturated conditions. Further enhancements include the reset of the backoff stage when the transmission queue is empty and to determine what is its impact on the overall performance of the protocol.

%Further enhancements to CSMA/ECA include the reset of the backoff stage when the transmission queue is empty as well as performance tests in non-saturated conditions. This introduces new challenges related to convergence time and delay which need to be leveraged.

Also, future development will be focused on implementing CSMA/ECA in cheap commodity hardware~\cite{WMP}. Doing so will open the door for evaluation under more realistic scenarios as well as provide insight on different communication aspects, for example those regarding channel errors, delay, synchronization, coexistence with other access protocols and real network traffic. 

%Nevertheless, this is not a trivial task given that in requires flexible network interface hardware which is not commonly provided by manufacturers. 

\section*{Acknowledgments}
The authors would like to thank Azadeh Faridi for her insightful comments and contributions.