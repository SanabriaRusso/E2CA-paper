% \documentclass[a4paper,journal]{IEEEtran}
\documentclass[conference]{IEEEtran}
%% INFOCOM 2013 addition:
\makeatletter
\def\ps@headings{%
\def\@oddhead{\mbox{}\scriptsize\rightmark \hfil \thepage}%
\def\@evenhead{\scriptsize\thepage \hfil \leftmark\mbox{}}%
\def\@oddfoot{}%
\def\@evenfoot{}}
\makeatother
\pagestyle{headings}
\usepackage{psfrag}

% \usepackage{auto-pst-pdf}

\usepackage[utf8]{inputenc}
\usepackage{graphicx}
\usepackage{float}
\usepackage{color, colortbl}
\usepackage{xcolor}
\usepackage{array}
\usepackage{multirow}
\usepackage{footnote}
\usepackage{cite}
%The below is used to add notes to tables without disrupting the IEEEtran format
\usepackage{threeparttable}

% Disable below if wanting to comply exclusively to conference mode of IEEEtran
% \IEEEoverridecommandlockouts

\makesavenoteenv{tabular}

%Ignores \vbox errors below the level of 10000
% \vbadness=10000



\begin{document}
%opening
 \title{Fairness in Collision-free WLANs}


%A more simple output, useful when involving people from different affiliations
  \author{
      \IEEEauthorblockN{Luis Sanabria-Russo\IEEEauthorrefmark{0}, Jaume Barcelo\IEEEauthorrefmark{0}, Boris Bellalta\IEEEauthorrefmark{0}}\\
      \IEEEauthorblockA{\IEEEauthorrefmark{0}Universitat Pompeu Fabra, Barcelona, Spain
      \\\{luis.sanabria, jaume.barcelo, boris.bellalta\}@upf.edu}
  }

%This is the style of three columns, as indicated in IEEEtran
% \author{\IEEEauthorblockN{Luis Sanabria-Russo}
%  \IEEEauthorblockA{Department of Information\\
%  and Communications Technologies\\
%  Universitat Pompeu Fabra\\
%  Barcelona, Spain\\
%  Email: luis.sanabria@upf.edu}
%  \and
%  \IEEEauthorblockN{Jaume Barcelo}
%  \IEEEauthorblockA{Department of Information\\
%  and Communications Technologies\\
%  Universitat Pompeu Fabra\\
%  Barcelona, Spain\\
%  Email: cristina.cano@upf.edu}
%  \and
%  \IEEEauthorblockN{Boris Bellalta}
%  \IEEEauthorblockA{Department of Information\\
%  and Communications Technologies\\
%  Universitat Pompeu Fabra\\
%  Barcelona, Spain\\
%  Email: boris.bellalta@upf.edu}}


\maketitle

\begin{abstract}

\boldmath It is possible to achieve a collision-free state implementing Carrier Sense Multiple Access with Enhanced Collision Avoidance (CSMA/ECA). It differs from CSMA/CA in choosing a deterministic (instead of random) backoff after successful transmissions. Also in CSMA/ECA, contenders keep the increased length of the contention window even after a successful transmission, what results in an uneven distribution of the channel access time. This fairness issue is assessed in this work by adjusting the number of packets each contender is allowed to transmit on each opportunity. Results show a totally distributed, collision-free and fair protocol capable of achieving higher levels of throughput than those of the conventional CSMA/CA.

%On this work, this fairness issue is assessed and it is revealed how by adjusting the number of packets transmitted on each opportunity. Results show a totally distributed, collision-free and fair protocol capable of achieving higher levels of throughput than those of the conventional CSMA/CA.

%Furthermore, the enhanced CSMA/E2CA has stickiness degrees, which refer to number deterministic backoffs used after each successful transmission and account for shorter convergence time towards a collision-free state. Implementing CSMA/E2CA in a totally distributed way revealed the unfair nature of the protocol. This abstract introduces the concept of Fair Share as a way to leverage this issue, which consist of adapting the number of packets to be transmitted accordingly with the backoff stage of each node. Results show a totally distributed, collision-free and fair protocol capable of achieving higher levels of throughput than those of the conventional CSMA/CA.

\end{abstract}

\begin{IEEEkeywords}
Wireless, MAC, Collision-free, CSMA/ECA.
\end{IEEEkeywords}

\section{Introduction} \label{introduction}
  Carrier Sense Multiple Access with Enhanced Collision Avoidance (CSMA/ECA)~\cite{CSMA_ECA} achieves less collisions and outperforms CSMA/CA in most typical scenarios. This is done by picking a deterministic backoff after each successful transmission. Its evolution, CSMA/E2CA introduces stickiness in the process in order to shorten the convergence time towards a collision-free state by setting a number of occasions a deterministic backoff is used after each successful transmission.

Stickiness can reduce the convergence time by orders of magnitude when the number of contenders $\eta$ is less or equal than the expectation of the random backoff used in CSMA/CA, $C$. The same constraint is valid for improvements in throughput, as can be appreciated in Figure~\ref{fig:throughput}.

\begin{figure}[htbp]
  \centering
%   \psfrag{psfrag1}[Bc][Bc][0.9]{$\eta$}
  \includegraphics[width=0.7\linewidth, angle = -90]{figures/throughput/throughput.eps}
  \caption{Throughput and how it is affected when $\eta \geq C$
  \label{fig:throughput}}
\end{figure}

In Figure~\ref{fig:throughput}, when $\eta \geq C$ the CSMA/E2CA system is overcrowded with contenders and the collision-free state is compromised. As more contenders are introduced, the system behavior tends to be more like CSMA/CA.

In this work, a fully-distributed version of CSMA/E2CA is presented and the throughput issue when $\eta \geq C$ is assessed using Fair Share.

\section{A descentralized and fair CSMA/ECA} \label{csmae2ca}
  In an overcrowded CSMA/ECA ($N>N_{max}$), nodes will double $CW(k)$ after collisions and reset it ($CW(k)=CW_{min}$) upon each transmission success, augmenting the collision probability. This accounts for the throughput reduction in Figure~\ref{fig:throughput}. To make it possible to achieve the collision-free state when $N>N_{max}$, we propose that nodes in CSMA/ECA do not reset $CW(k)$ after successful transmissions. This is called \emph{resiliency} from here forward.

Resiliency forces nodes to \emph{stick} to the value of the current backoff stage, $k$; resulting in a larger $CW(k)$. This measure leads to a collision-free state while $N\leq N_{max}$.

Having a greater collision-free constraint ($N_{max}$) means that more nodes are able to achieve a collision-free state. Nevertheless, in a $N\leq N_{max}$ scenario, contenders may have different deterministic backoff counters which provoke some nodes to access the channel more often than others. This fairness issue is averted with \emph{fair-share}.

Fair-share consist on allowing each contender to send $2^{k}$ packets every time its backoff expires ($B=0$), making sure that contenders with longer backoff are compensated proportionally.

Figure~\ref{fig:fairShare}, depicts how CSMA/ECA with resiliency and fair-share achieves greater throughput than CSMA/CA, maintaining a collision-free state and being fair (Jain's Fairness Index~\cite{JFI}~(JFI) equals 1) for any number of contenders.

\begin{figure}[htbp]
  \centering
  \includegraphics[width=0.7\linewidth, angle = -90]{figures/throughput/CSMA-E2CA_w_fairShare.eps}
  \caption{Throughput and Jain's Fairness Index when implementing fair-share in CSMA/ECA
  \label{fig:fairShare}}
\end{figure}

The concept fair-share, was first introduced by Fang et al. Available at SpringerLink~\cite{L_MAC2}. This work evaluates the performance of CSMA/ECA when implementing the concept in a customized C++ simulator.

%In Figure~\ref{fig:fairShare} it is appreciated through the estimation of the Jain's Fairness Index~\cite{JFI} that by implementing Fair Share every contender receives almost the same service time, therefore the system is considered fair.

\subsection*{Evaluation}
CSMA/ECA preserves backward compatibility with CSMA/CA (details in~\cite{CSMA_ECA, HE}), which is paramount for the coexistence and progressive adoption of the protocol. Many other performance evaluations, like a semi-analytical framework modelling the enhanced collision avoidance mechanism and comparing it with other access schemes (like Basic Access and RTS/CTS), are provided in~\cite{E2CA_performance}. Nevertheless, to the best of our knowledge this is the first evaluation of resilience and fair-share in CSMA/ECA.

Implementation is performed on a customized version of the COST~\cite{COST}~simulator. The system was set to be under saturation (nodes always have packets to transmit) during a period of a hundred thousand seconds at a maximum throughput of $11$Mbps. The number of contenders ranges from $2$ to $50$. Further MAC-related parameters can be found under~\emph{stats/stats.h}~in~\cite{sim:parameters}; as well as the code for the whole CSMA/ECA implementation.

Figure~\ref{fig:throughput} and Figure~\ref{fig:fairShare} are results derived from the evaluation platform.

\section{Conclusions} \label{conclusions}
  \input{conclusions}
  
\bibliographystyle{Classes/IEEEtran}
\bibliography{IEEEabrv,ref}
  
\end{document}

